\documentclass[paper=a4, fontsize=11pt]{scrartcl} 

\usepackage[T1]{fontenc} 
\usepackage[english]{babel}
\usepackage{amsmath,amsfonts,amsthm}
\usepackage{lipsum}
\usepackage{graphicx}
\usepackage{float}
  \floatplacement{figure}{H}
  \floatplacement{table}{H}
\usepackage{sectsty} 
\allsectionsfont{\centering \normalfont\scshape} 
\usepackage{fancyhdr} % Custom headers and footers
\pagestyle{fancyplain} % Makes all pages in the document conform to the custom headers and footers
\fancyhead{} % No page header - if you want one, create it in the same way as the footers below
\fancyfoot[L]{} % Empty left footer
\fancyfoot[C]{} % Empty center footer
\fancyfoot[R]{\thepage} % Page numbering for right footer
\renewcommand{\headrulewidth}{0pt} % Remove header underlines
\renewcommand{\footrulewidth}{0pt} % Remove footer underlines
\setlength{\headheight}{13.6pt} % Customize the height of the header

\numberwithin{equation}{section} % Number equations within sections (i.e. 1.1, 1.2, 2.1, 2.2 instead of 1, 2, 3, 4)
\numberwithin{figure}{section} % Number figures within sections (i.e. 1.1, 1.2, 2.1, 2.2 instead of 1, 2, 3, 4)
\numberwithin{table}{section} % Number tables within sections (i.e. 1.1, 1.2, 2.1, 2.2 instead of 1, 2, 3, 4)

\setlength\parindent{0pt} % Removes all indentation from paragraphs 

% -------- tambahan -------------
% -------------------------------
\usepackage[labelformat=empty]{caption} % remove caption number in figure
\usepackage{color}
\usepackage{listings}
\lstset{ %
language=bash,                % choose the language of the code
basicstyle=\footnotesize,       % the size of the fonts that are used for the code
numbers=left,                   % where to put the line-numbers
numberstyle=\footnotesize,      % the size of the fonts that are used for the line-numbers
stepnumber=1,                   % the step between two line-numbers. If it is 1 each line will be numbered
numbersep=5pt,                  % how far the line-numbers are from the code
backgroundcolor=\color{white},  % choose the background color. You must add \usepackage{color}
showspaces=false,               % show spaces adding particular underscores
showstringspaces=false,         % underline spaces within strings
showtabs=false,                 % show tabs within strings adding particular underscores
frame=single,           % adds a frame around the code
tabsize=2,          % sets default tabsize to 2 spaces
captionpos=b,           % sets the caption-position to bottom
breaklines=true,        % sets automatic line breaking
breakatwhitespace=false,    % sets if automatic breaks should only happen at whitespace
escapeinside={\%*}{*)}          % if you want to add a comment within your code
}

%----------------------------------------------------------------------------------------
%	TITLE SECTION
%----------------------------------------------------------------------------------------

\newcommand{\horrule}[1]{\rule{\linewidth}{#1}} % Create horizontal rule command with 1 argument of height

\title{	
\normalfont \normalsize 
\textsc{Computational Science - ITB} \\ [25pt] % Your university, school and/or department name(s)
\horrule{0.5pt} \\[0.4cm] % Thin top horizontal rule
\small Algoritma dan Desain Perangkat Lunak \\ 
\large Desain Sistem Pembuatan SIM (Surat Izin Mengemudi): \textit{Waterfall Model}\\  % The assignment title
\horrule{1.5pt} \\[0.5cm] % Thick bottom horizontal rule
}
\author{\small Febrie Ahmad A. || 20912008 \\ \small Ridlo W. Wibowo || 20912009} % Your name
\date{\normalsize\today} % Today's date or a custom date

% -------------------------------------------------------------------------------
\begin{document}

\maketitle % Print the title

%\section{Requirements specification}
%\subsection{abc}

%\large \textbf{Problem.}
%Buatlah program untuk solusi Parabolic Partial Differential Equations (PDE),\\
%- \textit{Forward Difference Method} (FTCS)\\ 
%- \textit{Backward Difference Method} (BTCS)\\
%- \textit{Crank-Nicolson Method} (C-N)\\
%Lalu tentukan solusi untuk PDE:\\
%\begin{equation}
%\frac{\partial u}{\partial t} = \frac{\partial ^{2}u}{\partial x^{2}}
%\end{equation}
%untuk $0 < x < \pi$ dan $t > 0$ dengan,
%\begin{itemize}
%\item syarat Batas:\\
%$u(0, t) = u(\pi, t) = 0.0$ untuk $t > 0$\\
%\item syarat Awal:\\
%$u(x,0) = \sin(x)$ untuk  $0 \leq x \leq \pi$\\ 
%\end{itemize}
%dan bandingkan hasilnya untuk $t = 0.5$.\\
%(solusi eksak $u(x, t) = e^{-t}\sin{x}$)

%\newpage
%\large \textbf{Result.}
%Dari penurunan dan algoritma yang diberikan di buku \textit{Numerical Analysis} oleh Richard L.Burden dan J. Douglas Faires, lalu dapat diterapkan untuk membuat program penyelesaian masalah PDE. Program yang telah dibuat terlampir di akhir (\textit{ftcs.cpp}, \textit{btcs.cpp}, \textit{CN.cpp}).
%
%\begin{table}[ht]
%\begin{tabular}{c c c c}
%\hline
%$x_{i}$ & $u_{(x_{i}, 0.5)}$ & $w_{(x_{i}, 0.5)}$ & $\vert u - w \vert _{(x_{i}, 0.5)}$  \\ [0.5ex]
%\hline 
% 1 &  2 &  3 & 4 \\
% &  &  & \\
% &  &  & \\
% &  &  & \\
% &  &  & \\
% &  &  & \\
% &  &  & \\
% &  &  & \\
% &  &  & \\
% &  &  & \\
% &  &  & \\ [1ex]
%\hline 
%\end{tabular}
%\end{table}
%
%\begin{figure}
%	\centering
%	\includegraphics[width=0.8\textwidth]
%		{plot.png}
%	\caption{Plot hasil akhir ($t = 0.5$).}
%\end{figure}
%
%\begin{figure}
%	\centering
%	\includegraphics[width=0.8\textwidth]
%		{zoom.png}
%	\caption{Diperbesar disekitar $x = \frac{\pi}{2}$ ($t = 0.5$).}
%\end{figure}
%


\end{document}














